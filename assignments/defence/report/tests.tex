\subsection{Motivation}
Tests have been an important phase in our whole development. In order to insure
that our implementation was still working after any change, we made a lot of
regression tests. 

It means that, as soon as we added a new feature which was compiling, we were
able to check if it didn't generate any bugs or if it broke something we made before.

The idea was to create these tests in parallel with the implementation of new
parts of the kernel. Thus, we tried to test everything we implemented on top of
the given Nachos kernel.

\subsection{The way it works}
\begin{itemize}
    \item $*.c$ files, in test directory, contains some programs which helped us
        test user-level features and system calls
    \item $x.desc$ files contains descriptions about the matching $x.sh$ scripts
    \item $x.sh$ are bash scripts calling either C programs or launching directly
        one nachos command line option
\end{itemize}


\subsection{How to run them}
To run all the tests, just compile the kernel then enter the following command :

\textbf{make regression}

It will run tests for all steps. Those which are marked as "ok" didn't raised
any error, the others did.\\

In order to not be lost while tests are running, we specified the step we were
working on at the beginning of the test name. For instance, step5 tests are
related to file-system.\\

To create our tests we proceeded the following way :
\begin{itemize}
    \item First we create some basic test, as unitary as possible trying to
        cover each execution flow (including error handling). 
    \item After we create more complex tests using execution scenarios.
\end{itemize}

We finally made 132 tests for all nachos parts. 

\subsection{Examples}
\subsubsection{Consumer/Producer}
We implemented a producer/consumer test in user-mode using circular buffer.
We used two semaphores and one mutex. The mutex protect the critical section
where buffer is manipulated. The two semaphores are used as counter :
\begin{itemize}
    \item One for the number of empty slot inside buffer
    \item One for the number of currently available items inside buffer
\end{itemize}

We do this with five producers and five consumers with buffer of size 3.

\subsubsection{The tiny shell}
A shell was implemented with jobs support. You can launch programs using
ForkExec.
These programs can be launch either in foreground or background using \& at
the end of the command line.

The "jobs" command list all currently background running processes.

The "fg" command is used to switch a background process to foreground.

These tests allow to check ForkExec and WaitPid syscalls as well as process
management.

\subsubsection{Stress process}
In the aim of testing our implementation limits on processes and threads, we
tried to create 150 processes, each of them with 40 threads.
It allows us to check if Nachos does not misbehave and handle limits properly.
Thus, running this test, we see that when the bound is reach, bad request for
new process/threads creation are rejected.

%\begin{itemize}
%	\item GetChar() and PutChar()
%		\begin{itemize}
%		\item Basic Input Output character test
%		\item {\bf Test for character EOF}
%		\item {\bf Test for ASCII 255}
%		\end{itemize}
%	\item GetString() and PutString()
%		\begin{itemize}
%		\item Basic Input Output string test
%		\item {\bf Test for empty string}
%		\item {\bf Test for string EOF}
%		\item {\bf Test Maximum String}
%		\end{itemize}
%	\item GetInt() and PutInt()
%		\begin{itemize}
%		\item Basic Input Output Integer
%		\item {\bf Test Int EOF}
%		\item {\bf Test Int Everflow}
%		\end{itemize}
%\end{itemize} 
%
%\subsection{Part 3}
%
%3rd part implement threads and we provided such test the interesting one will mark as bold: 
%
%\begin{itemize}
%\item {\bf step3-multiple-join}
%\item {\bf step3-synchconsole-synch-put}
%\item {\bf step3-synchconsole-synch-rw}
%\item step3-synchconsole-synch
%\item {\bf step3-test-exit-delete-chilren}
%\item {\bf step3-test-recursive-threads-kill}
%\item {\bf step3-test-recursive-threads-simple}
%\item step3-threadArg
%\item step3-threadcreate
%\item step3-thread-exit-code-wait-too-late
%\item step3-thread-exit-delete-children
%\item step3-thread-exit
%\item {\bf step3-thread-join-after-join}
%\item step3-threadjoinerror
%\item {\bf step3-threadJoinMax}
%\item {\bf step3-threadJoinMultiple}
%\item step3-threadjoin
%\item step3-threadJoinSimple
%\item step3-thread-main-userthreadexit
%\item {\bf step3-thread-max-limit}
%\item {\bf step3-thread-Multiple-Kill-Create}
%\item {\bf step3-threadProdCons}
%\item step3-thread-return-code
%\item step3-threadSemaphore
%\item step3-thread-userthreadexit-function
%\item {\bf step3-use-destroyed-semaphore}
%\end{itemize} 
%
%\subsection{Part 4}
%
%This parts implement Memory Management and Process. We provided such test the interesting one will mark as bold :
%\begin{itemize}
%\item step4-fork-unknow-program
%\item {\bf step4-heap-alloc-free-behavior}
%\item {\bf step4-malloc-bad-free}
%\item step4-malloc-concurrent
%\item step4-malloc-free-multiple
%\item {\bf step4-malloc-just-fit}
%\item {\bf step4-malloc-multiple-process}
%\item step4-malloc-reuse-memory
%\item step4-malloc-simple 
%\item step4-malloc-will-fail
%\item {\bf step4-multiple-ForkExec}
%\item {\bf step4-Multiple-ForkExec-Waitpid} 
%\item step4-process-preempt
%\item step4-stress-process-thread
%\item step4-thread-Join-0
%\item {\bf step4-tiny-shell-test-extend}
%\item step4-tiny-shell-test
%\item {\bf step4-trigger-page-fault-multiple-process}
%\item step4-trigger-page-fault
%\item step4-userpages0
%\item step4-waitpid-return
%\end{itemize} 
%
%\subsection{Part 5}
%
%This parts implement FileSystem We provided such test the interesting one will mark as bold :
%\begin{itemize}
%\item step5-absolute-path
%\item {\bf step5-change-directory-kernel-one-thread}
%\item {\bf step5-change-directory-kernel-two-thread}
%\item step5-change-directory-simple
%\item step5-change-directory-thread
%\item {\bf step5-cp-verified}
%\item step5-create-directory-dot-dot-name
%\item step5-create-dot-name-file
%\item step5-create-dot-name
%\item step5-create-file-as-directory
%\item {\bf step5-create-file-bad-name}
%\item step5-create-file-dot-dot-name
%\item step5-create-file
%\item step5-create
%\item step5-directory-limit
%\item step5-directory-limit-third
%\item {\bf step5-directory-limit-with-files}
%\item {\bf step5-file-create-bad-name-directory}
%\item step5-file-create-directory-relative
%\item step5-file-create-directory
%\item step5-file-create-existing-directory
%\item {\bf step5-file-create-file-exist-directory}
%\item step5-file-create-file-relative
%\item step5-file-listing
%\item step5-fill-disk
%\item step5-listing-directory-one-level-second
%\item step5-listing-directory-one-level
%\item step5-listing-directory-relative
%\item step5-listing-directory-simple
%\item {\bf step5-max-file-open-fork}
%\item step5-max-file-open-simple
%\item step5-max-file-open-thread
%\item step5-multiple-file-open
%\item {step5-open-file-table-rw}
%\item step5-open-same-file-process
%\item step5-recursive-listing
%\item {\bf step5-remove-existing-empty-directory-first}
%\item {\bf step5-remove-existing-empty-directory-last}
%\item {\bf step5-remove-existing-empty-directory-middle}
%\item step5-remove-file
%\item step5-remove-non-existing-directory
%\item step5-remove-opened-file
%\item {\bf step5-remove-recreate-directory}
%\item step5-remove-relative-path
%\item step5-remove-root
%\item {\bf step5-rw-concurrent-read}
%\item step5-seek
%\item step5-thread-close-read
%\item step5-threads-open-write-read-close
%\item step5-threads-write
%\item {\bf step5-too-big-file}
%\item {\bf step5-too-large-file}
%\item {\bf step5-two-thread-opening-file}
%\end{itemize} 
%
%\subsection{Part 6}
%
%This parts implement Network. We provided such test the interesting one will mark as bold :
%\begin{itemize}
%\item {\bf step6-multiple-listen-same-port}
%\item step6-receive-message-waiting
%\item step6-send-message-not-connected
%\item step6-simple-listen-connect-accept
%\item step6-test-accept-failed
%\item step6-test-incomplete-acknowledgement
%\item {\bf step6-test-big-message}
%\item {\bf step6-test-multithread}
%\end{itemize} 
