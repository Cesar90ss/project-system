%%%%%%%%%%%%%%%%%%%%%%%%%%%%%%%%%%%%%%%%%%%%%%%%%%%%
%					Example of Report in LaTeX format                    					   %
%																			   %
%	File: organization.tex	 															   %
%	Author: ???																   %
%																			   %
%																			   %
%%%%%%%%%%%%%%%%%%%%%%%%%%%%%%%%%%%%%%%%%%%%%%%%%%%%

\documentclass[a4paper,10pt]{article}

\usepackage{graphicx}
\usepackage{fullpage}
\usepackage{titling}
\usepackage{listings}
\lstset{%
  basicstyle=\scriptsize\sffamily,%
  commentstyle=\footnotesize\ttfamily,%
  frameround=trBL,
  frame=single,
  breaklines=true,
  showstringspaces=false,
  numbers=left,
  numberstyle=\tiny,
  numbersep=10pt,
  keywordstyle=\bf
}
\newcommand{\subtitle}[1]{%
  \posttitle{%
    \par\end{center}
    \begin{center}\large#1\end{center}
    \vskip0.5em}%
}



\title{Nachos Project : Report}
\subtitle{Part 5 : Organisation}

\begin{document}


\maketitle

\section{User Test}

Test is an important phase in our whole development system which ensure all of implementation is work well and 
not contain any bug. We have devided our test in every step and we make sure finished our implementation with 
complete and compiled test and make sure no more bug on previous implementation. For all the test we provide
both manual test in ./test folder and written in c and automatic test in ./regression folder and written in bash 
this test will automatically compiled return "ok" if success and return "error" if the test failed.Furthermore 
automatic test (regression test) we provided due to monitor our particular function of system we developed.
The regression test we made consist of test file and description about the detail test we made. Moreover
every manual test we made will be basic test to generate relating automatic test. And what 
the most benefit of creating regression test is for check if during develop and modified the file that has 
dependencies to another test, the regression test will give us rapid result that monitor every function work well
and the new function we have create did not broke our past test and implementation. We decide to make test case 
on the normal case and the error case to catch every possible condition of the system compilation. We only provide
test for user level code that give good information as feedback to the user. 


\subsection{Part 1 to 2}

In the 1st step there is no particular test we need just make sure that NachOS work well in our machine. We run 
basic test input and output how to track the code, how to handling error and understanding basic command and all 
tools are needed to develop these system. 

For the 2nd step we implement basic Input/Output for the system. There is 6 basic Input/Output have to implement.
There are 6 basic systemcall relating to basic Input/Output : GetChar(), PutChar(), GetString(), PutString(), 
GetInt() and PutInt(). In this step we provided complete test for each syscalls basic called on the syscalls. And 
particular detail test will describe in following part :

\begin{itemize}
	\item GetChar() and PutChar()
		\begin{itemize}
		\item Basic Input Output character test
		\item {\bf Test for character EOF}
		\item {\bf Test for ASCII 255}
		\end{itemize}
	\item GetString() and PutString()
		\begin{itemize}
		\item Basic Input Output string test
		\item {\bf Test for empty string}
		\item {\bf Test for string EOF}
		\item {\bf Test Maximum String}
		\end{itemize}
	\item GetInt() and PutInt()
		\begin{itemize}
		\item Basic Input Output Integer
		\item {\bf Test Int EOF}
		\item {\bf Test Int Everflow}
		\end{itemize}
\end{itemize} 

\subsection{Part 3}

3rd part implement threads and we provided such test the interesting one will mark as bold: 

\begin{itemize}
\item {\bf step3-multiple-join}
\item {\bf step3-synchconsole-synch-put}
\item {\bf step3-synchconsole-synch-rw}
\item step3-synchconsole-synch
\item {\bf step3-test-exit-delete-chilren}
\item {\bf step3-test-recursive-threads-kill}
\item {\bf step3-test-recursive-threads-simple}
\item step3-threadArg
\item step3-threadcreate
\item step3-thread-exit-code-wait-too-late
\item step3-thread-exit-delete-children
\item step3-thread-exit
\item {\bf step3-thread-join-after-join}
\item step3-threadjoinerror
\item {\bf step3-threadJoinMax}
\item {\bf step3-threadJoinMultiple}
\item step3-threadjoin
\item step3-threadJoinSimple
\item step3-thread-main-userthreadexit
\item {\bf step3-thread-max-limit}
\item {\bf step3-thread-Multiple-Kill-Create}
\item {\bf step3-threadProdCons}
\item step3-thread-return-code
\item step3-threadSemaphore
\item step3-thread-userthreadexit-function
\item {\bf step3-use-destroyed-semaphore}
\end{itemize} 

\subsection{Part 4}

This parts implement Memory Management and Process. We provided such test the interesting one will mark as bold :
\begin{itemize}
\item step4-fork-unknow-program
\item {\bf step4-heap-alloc-free-behavior}
\item {\bf step4-malloc-bad-free}
\item step4-malloc-concurrent
\item step4-malloc-free-multiple
\item {\bf step4-malloc-just-fit}
\item {\bf step4-malloc-multiple-process}
\item step4-malloc-reuse-memory
\item step4-malloc-simple 
\item step4-malloc-will-fail
\item {\bf step4-multiple-ForkExec}
\item {\bf step4-Multiple-ForkExec-Waitpid} 
\item step4-process-preempt
\item step4-stress-process-thread
\item step4-thread-Join-0
\item {\bf step4-tiny-shell-test-extend}
\item step4-tiny-shell-test
\item {\bf step4-trigger-page-fault-multiple-process}
\item step4-trigger-page-fault
\item step4-userpages0
\item step4-waitpid-return
\end{itemize} 

\subsection{Part 5}

This parts implement FileSystem We provided such test the interesting one will mark as bold :
\begin{itemize}
\item step5-absolute-path
\item {\bf step5-change-directory-kernel-one-thread}
\item {\bf step5-change-directory-kernel-two-thread}
\item step5-change-directory-simple
\item step5-change-directory-thread
\item {\bf step5-cp-verified}
\item step5-create-directory-dot-dot-name
\item step5-create-dot-name-file
\item step5-create-dot-name
\item step5-create-file-as-directory
\item {\bf step5-create-file-bad-name}
\item step5-create-file-dot-dot-name
\item step5-create-file
\item step5-create
\item step5-directory-limit
\item step5-directory-limit-third
\item {\bf step5-directory-limit-with-files}
\item {\bf step5-file-create-bad-name-directory}
\item step5-file-create-directory-relative
\item step5-file-create-directory
\item step5-file-create-existing-directory
\item {\bf step5-file-create-file-exist-directory}
\item step5-file-create-file-relative
\item step5-file-listing
\item step5-fill-disk
\item step5-listing-directory-one-level-second
\item step5-listing-directory-one-level
\item step5-listing-directory-relative
\item step5-listing-directory-simple
\item {\bf step5-max-file-open-fork}
\item step5-max-file-open-simple
\item step5-max-file-open-thread
\item step5-multiple-file-open
\item {step5-open-file-table-rw}
\item step5-open-same-file-process
\item step5-recursive-listing
\item {\bf step5-remove-existing-empty-directory-first}
\item {\bf step5-remove-existing-empty-directory-last}
\item {\bf step5-remove-existing-empty-directory-middle}
\item step5-remove-file
\item step5-remove-non-existing-directory
\item step5-remove-opened-file
\item {\bf step5-remove-recreate-directory}
\item step5-remove-relative-path
\item step5-remove-root
\item {\bf step5-rw-concurrent-read}
\item step5-seek
\item step5-thread-close-read
\item step5-threads-open-write-read-close
\item step5-threads-write
\item {\bf step5-too-big-file}
\item {\bf step5-too-large-file}
\item {\bf step5-two-thread-opening-file}
\end{itemize} 

\subsection{Part 6}

This parts implement Network. We provided such test the interesting one will mark as bold :
\begin{itemize}
\item {\bf step6-multiple-listen-same-port}
\item step6-receive-message-waiting
\item step6-send-message-not-connected
\item step6-simple-listen-connect-accept
\item step6-test-accept-failed
\item step6-test-incomplete-acknowledgement
\item {\bf step6-test-big-message}
\item {\bf step6-test-multithread}
\end{itemize} 

\end{document}