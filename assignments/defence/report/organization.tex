\subsection{Implementation}
We decided to work everyday at the university so as to make communication
between members of the group easier. Allowing us to solve bug faster, avoiding
misunderstanding of the specification and cutting the work dynamically (reevaluating
planning in real time when needed to).\\
For each step, before beginning anything, we all read and understood the
subject. Then we discussed with each other, establishing a "todo list" and finally we
 distributed the work.\\ 
During the implementation, we often made summary of our work to explain the
implementation of every part of the code to the rest of the team.\\
To make the group work easier we used a git repository. This allowed us to
share our work more efficiently. This tool can manage code and handle merges automatically
or at least make them easier. Thus, we were able to share the code immediately
when needed.\\

\subsection{Validation}
Following the principle of the test driven development, a large set of automated
regression tests were developed. It was used to validate the code or point out
one or multiple bugs/regressions in a fast and easily understandable way. \\
A failure on a test will indicate what the test was about, thus allowing us to
point out the error easily. And by doing the test before or during the
implementation this automated test can allow to easily monitor the state and
the advancement of the part currently being implemented.\\
Indeed during the beginning of the implementation, one of the members would
implement the tests while the other began coding which allows to have tests
ready as soon as the program compiles.\\
Valgrind has also been extensively used to remove memory leaks and wrong
memory access. Thus, suppressing a significant number of possible bugs in the
program. The program should now be free of any illegal memory accesses and memory
leaks.

\subsection{Part 1 to 4}

For the parts 1 to 4, we followed the subject and all worked on the same part.
We usually split into two teams and divided the work  whole step into
multiple parts. In these teams, we practiced extreme programming so as to
compensate for the fact that all the work could not easily be split between
each member. Thus, making the debugging phase faster. During the complicated
phases, having two people working on the same problem often allowed us to
solve it faster and to avoid many bugs in the code that a single person would
not notice while coding.\\
Also, when working in pair (or triple) the tests were not done by the person
who was currently coding. Thus they were not made with the weakness of the
program in mind and tested uniformly the program without forgetting some part
or another.

\subsection{Part 5 and 6}

The parts 5 and 6 were independent, thus, we decided to make them in parallel.
The group was separated in two teams, each team working on one step.  For the
rest, the same philosophy as for the first parts was applied. That is, each
team worked following the extreme
programming principle.\\

The repository was split in branches in order to make the work easier for both
team. Thus, we were able to work on the same files without causing neither, unnecessary
conflicts, nor possibly bugs that would disturb the team not causing them. 

We used tags to mark the last commit of a given step.

\subsection{Last two days}

On the last days, one person took care of the merging of the two branches. The
others made a "todo List" for the report and presentation, then everyone took
one part of it to prepare. 
