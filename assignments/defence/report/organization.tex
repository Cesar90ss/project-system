\subsection{Implementation}
First we decided to work everyday at the university so as to make communication
between members of the group easier.  Allowing us to solve bug faster, avoid
derivation from the specification and cut the wok dynamically(reevaluate
planning in real time when need be).\\
For each step, before beginning anything, we would all read and understand the
subject. Then discuss with each other, establish a "todo list" and then we
would distribute the work.\\ Then, during the implementation, we would often
make summary of our work to explain the implementation of every part of the
code to the rest of the group.\\
To make the group work easier we used a git repository. This allowed us to
share our work more efficiently as code this tool can manage code merges
automatically or at least make them easier. And allow us to work even when not
at the university and still be able to share the code immediately if need be.\\

\subsection{Validation}
Following the principle of the test driven development a large set of automated
regression tests were developed.  It was used to validate the code or point out
one or multiple bug / regressions in it in a comfortable, fast way and easily
understandable way. \\
A failure on a test will indicate what the test was about, thus allowing us to
point out the error easily.And by doing the test before or during the
implementation this automated tests can allow to easily monitor the state and
the advancement of the part currently being implemented.\\
Indeed During the beginning of the  implementation, one of the members would
implement the tests while the other began coding which allows to have tests
ready as soon as the program compiles. \\
Valgrind has also been extensively used to remove memory leaks and unsafe
memory access. Thus suppressing a significant number of possible bugs in the
program. The program should now be free of any illegal memory access and memory
leak.

\subsection{Part 1 to 4}

For the parts 1 to 4, we followed the subject and all worked on the same part.
We usually split into two teams.  Splitting the works of the whole step into
two main parts. In this teams, we practiced extreme programing so as to
compensate for the fact that all the work could not easily be split between
each member. Thus making the debugging phase faster. During the complicated
phases, having two persons working on the same problem often allowed us to
solve it faster and to avoid many bugs in the code that a single person would
not notice while coding.\\
Also, when working in pair (or triple) the tests were not done by the person
who was currently coding, thus they were not made with the weakness of the
program in mind and tested uniformly the program without sparing some part or
another.

\subsection{Part 5 and 6}

The parts 5 and 6 were done in parallel. The group was separated in two teams,
each team working on one step.  For the rest, the same philosophy as for the
first parts was applied. That is, each team worked following the extreme
programing principle.\\
The git repository was split in two working branches to make the work easier
for both team as the two part were independent but required to work in the same
files thus causing unnecessary conflicts and possibly bugs that would disturb
the team not causing them. We used tags to separate these two branches.

\subsection{Last two days}

On the last days, one person took care of the merging of the two branches. The
other made a "todo List" for the report and presentation, then everyone took
one part of it to prepare. 
