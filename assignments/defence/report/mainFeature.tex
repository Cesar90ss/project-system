
\documentclass[a4paper,10pt]{article}
\usepackage{amsthm}
\usepackage{amsmath}
\usepackage{amsfonts}
\usepackage{mathtools}
\usepackage{wrapfig}
\usepackage{graphicx}
\usepackage{titling}
\usepackage{float}
\usepackage{caption}
\usepackage{subcaption}
\usepackage{listings}
\usepackage{fullpage}
\lstset{%
    basicstyle=\scriptsize\sffamily,%
    commentstyle=\footnotesize\ttfamily,%
    frameround=trBL,
    frame=single,
    breaklines=true,
    showstringspaces=false,
    numbers=left,
    numberstyle=\tiny,
    numbersep=10pt,
    keywordstyle=\bf
}
\newcommand{\subtitle}[1]{%
    \posttitle{%
        \par\end{center}
    \begin{center}\large#1\end{center}
\vskip0.5em}%
}

\newcommand{\CCscript}[2]{
    \begin{itemize}
        \item[]\lstinputlisting[caption=#2,label=#1,language=CC]{src/#1.cc}
    \end{itemize}
}
\lstdefinelanguage{codeoutput} { %this is the name that you are going to use when you want to use the formatting
    basicstyle=\ttfamily\scriptsize, %font family & size
}
\newcommand{\Oscript}[2]{
    \begin{itemize}
        \item[]\lstinputlisting[caption=#2,label=#1,language=codeoutput]{src/#1}
    \end{itemize}
}
\newcommand{\Cscript}[2]{
    \begin{itemize}
        \item[]\lstinputlisting[caption=#2,label=#1,language=CC]{src/#1.c}
    \end{itemize}
}

\setlength{\abovecaptionskip}{1pt plus 1pt minus 1pt} % Chosen fairly arbitrarily

\title{Nachos}
\subtitle{Project [MOSIG - M1]}
\author{Antoine Faravelon, Lucas Felix, Miratul Khusna Mufida, Hugo Guiroux, Simon Moura}
\date{\today}

\begin{document}

\section{Main Features Of The Program}


The nachos kernel offers most of the basic functionalities awaited from a kernel :\\
\begin{itemize}
\item synchronized Input/Output
\item User Multi Threading
\item Virtual memory and multiple processes running concurently
\item A file system
\item TCP/IP protocol
\end{itemize}

\subsection{A few interesting features}

First, all the the network functionalities have also been ported to the userspace. A user program can use sockets and 
transfer content other the network using the faclities offered by our nachos kernel. \\
We have also implemented the auto exit of the userthread so that, at the end of their execution, they do not return
into main code. And we can also catch their return values.\\
For this same threads, the kernel also give access to facilities to synchronize them (semaphore).\\
Processes have also been implemented with some managing functions. Indeed, though the kernel does not offer any 
processes hierarchy, a process can wait for the end of another one and then catch its exit value.\\
Also, in the userspace, we provide a minimal shell for use with our kernel. It provides a prompt and the ability
to launch program. Job management is also provided.\\
Finally, dynamic memory allocation has been implemented for the userspace. User can allocate \\dynamically on the heap 
instead of the stack.\\
After that, the user has access to a whole range of system calls that will be detailed in the following section.

\end{document}