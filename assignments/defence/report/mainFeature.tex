The nachos kernel offers most of the basic features awaited from a kernel :\\
\begin{itemize}
\item Synchronized Input/Output
\item User Multi Threading
\item Virtual memory
\item Concurrent multi-processes
\item A File System
\item TCP/IP like protocol
\end{itemize}

\subsection{Interesting features}

First, all the network features have been ported to the user space. A user
program can use sockets and transfer content through the network using the
means offered by our nachos kernel.\\

For these same threads, the kernel also give access to facilities to
synchronize them (semaphore).\\

Processes have also been implemented with some managing functions. Indeed,
though the kernel does not offer any processes hierarchy, a process can wait
for the end of another one and then catch its exit value.\\

In user space, we provide a minimal shell which gives access to a prompt and
the ability to launch programs with jobs management.\\

Finally, dynamic memory allocation has been implemented for the user space.
User can allocate dynamically on the heap instead of the stack.\\

After that, the user has access to a whole range of system calls that will be
detailed in the appendix. We consider our kernel fully-tested and we ensure no
memory leak or bad memory accesses.
