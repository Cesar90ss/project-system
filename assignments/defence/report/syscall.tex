\subsection{Syscalls}
\begin{description}
    \item [NAME] : \textbf{halt}
        \begin{itemize}
            \item SYNOPSIS : void Halt()
            \item DESCRIPTION :
                Halt is a system call which power off the system.
        \end{itemize}


    \item [NAME] : \textbf{Exit}
        \begin{itemize}
            \item SYSNOPSIS : void Exit(int status)
            \item DESCRIPTION :
                The exit syscall is used to quit the process. It will not shut down the
                machine unless there is no other process running.
        \end{itemize}

    \item [NAME] : \textbf{PutChar}
        \begin{itemize}
            \item SYSNOPSIS : void PutChar(char c)
            \item DESCRIPTION :
                PurtChar is a function which will write the character c to the console.
        \end{itemize}

    \item [NAME] : \textbf{GetChar}
        \begin{itemize}
            \item SYSNOPSIS : int GetChar()
            \item DESCRIPTION :
                GetChar is a function that read a character from the input buffer.
            \item RETURN :
                \begin{itemize}
                    \item The character itself or EOF if there is none in the buffer
                \end{itemize}
        \end{itemize}

    \item [NAME] : \textbf{PutInt}
        \begin{itemize}
            \item SYSNOPSIS : void PutInt(int i)
            \item DESCRIPTION :
                PutInt is a function which is used to write the integer i to the console.
        \end{itemize}

    \item [NAME] : \textbf{GetInt}
        \begin{itemize}
            \item SYSNOPSIS : int GetInt(int* p)
            \item DESCRIPTION :
                GetInt read an integer from the input buffer pointed out by p.
            \item RETURN :
                \begin{itemize}
                    \item 0 if there is no error
                    \item -1 when the input cannot be read as an int
                    \item -2 when the address p cannot be written by the caller
                \end{itemize}
        \end{itemize}

    \item [NAME] : \textbf{PutString}
        \begin{itemize}
            \item SYSNOPSIS : void PutString(const char s[])
            \item DESCRIPTION :
                PutString writes the string $s$ to the console. If the string given is longer than
                MAX\_STRING\_SIZE then the remaining part is not printed in the console.
        \end{itemize}

    \item [NAME] : \textbf{GetString}
        \begin{itemize}
            \item SYSNOPSIS : char *GetString(char *s, int n)
            \item DESCRIPTION :
                It reads at most $n-1$ characters in the console.
            \item RETURN :
                \begin{itemize}
                    \item $s$ if there is no error
                    \item otherwise return NULL (in case of error or EOF)
                \end{itemize}
        \end{itemize}

    \item [NAME] : \textbf{UserThreadCreate}
        \begin{itemize}
            \item SYSNOPSIS : int UserThreadCreate(void f(void *arg), void *arg)
            \item DESCRIPTION :
                This function create a new thread which will execute the function f with the
                argument arg.
            \item RETURN :
                \begin{itemize}
                    \item 0 on success
                    \item -1 if no space left for stack
                    \item -2 if MAX\_TOTAL\_THREADS (20 by default) has been reached
                \end{itemize}
        \end{itemize}


    \item [NAME] : \textbf{UserThreadExit}
        \begin{itemize}
            \item SYSNOPSIS : void UserThreadExit(void *ret)
            \item DESCRIPTION :
                This function is used to destroy the current thread and puts the return value in
                $ret$.
                When the main thread call UserThreadExit, other threads continue to
                run. The last thread to end will call Exit.
                When a thread function reach a return statement, it will be converted
                to this syscall with return value as argument.
        \end{itemize}

    \item [NAME] : \textbf{UserThreadJoin}
        \begin{itemize}
            \item SYSNOPSIS : int UserThreadJoin(int tid, void **retval)
            \item DESCRIPTION :
                This function is used to join another thread (eg : wait for the tread
                of tid $tid$ to terminate). If multiple threads tries to join on the same
                thread, only the first one will be able to join on it. The function
                will return an error for the others.
                If retval is not null, it contains the return value of exit thread,
                either by calling UserThreadExit or by reaching the end of thread function.
            \item RETURN :
                \begin{itemize}
                    \item 0 on success
                    \item -1 if bad tid
                    \item -2 if another thread is already joining on the same thread tid
                \end{itemize}
        \end{itemize}

    \item [NAME] : \textbf{UserSemaphoreCreate}
        \begin{itemize}
            \item SYSNOPSIS : int UserSemaphoreCreate(char* name, int value)
            \item DESCRIPTION :
                Initialize and return a semaphore id named "name" with an initial value "value".
                It do not create a semaphore with the id of a previously destroyed semaphore.
            \item RETURN :
                \begin{itemize}
                    \item Return the id of the semaphore freshly created
                \end{itemize}
        \end{itemize}

    \item [NAME] : \textbf{UserSemaphoreP}
        \begin{itemize}
            \item SYSNOPSIS : int UserSemaphoreP(int id)
            \item DESCRIPTION :
                Takes the lock on the semaphore pointed by id.
            \item RETURN :
                \begin{itemize}
                    \item 0 on success
                    \item -1 if error (semaphore does not exist)
                \end{itemize}
        \end{itemize}

    \item [NAME] : \textbf{UserSemaphoreV}
        \begin{itemize}
            \item SYSNOPSIS : int UserSemaphoreV(int id)
            \item DESCRIPTION :
                Release the lock (unlock) the semaphore pointed by id.
            \item RETURN :
                \begin{itemize}
                    \item 0 on success
                    \item -1 if error (semaphore does not exist)
                \end{itemize}
        \end{itemize}

    \item [NAME] : \textbf{UserSemaphoreDestroy}
        \begin{itemize}
            \item SYSNOPSIS : int UserSemaphoreDestroy(int id)
            \item DESCRIPTION :
                Destroy the semaphore pointed by id.
            \item RETURN :
                \begin{itemize}
                    \item 0 on success
                    \item -1 if error (semaphore does not exist)
                \end{itemize}
        \end{itemize}

    \item [NAME] : \textbf{AllocPageHeap}
        \begin{itemize}
            \item SYSNOPSIS : int AllocPageHeap()
            \item DESCRIPTION :
                AllocPageHeap asks for a new page on heap.
            \item RETURN :
                \begin{itemize}
                    \item -1 if no more page for heap
                    \item new page $addr$ otherwise
                \end{itemize}
        \end{itemize}

    \item [NAME] : \textbf{FreePageHeap}
        \begin{itemize}
            \item SYSNOPSIS : int FreePageHeap()
            \item DESCRIPTION :
                FreePageHeap gives back a new page for heap.
            \item RETURN :
                \begin{itemize}
                    \item The new heap top $addr$
                \end{itemize}
        \end{itemize}

    \item [NAME] : \textbf{ForkExec}
        \begin{itemize}
            \item SYSNOPSIS : unsigned int ForkExec(char *s)
            \item DESCRIPTION :
                ForkExec creates a new process that execute the program stated in the argument $s$.
            \item RETURN :
                \begin{itemize}
                    \item pid of the newly created process in case of creation success
                    \item -1 if more than MAX\_PROCESS processes have been created (by default 30)
                    \item -2 case of an invalid executable
                \end{itemize}
        \end{itemize}

    \item [NAME] : \textbf{Waitpid}
        \begin{itemize}
            \item SYSNOPSIS : int Waitpid(unsigned int pid, int *retval)
            \item DESCRIPTION :
                Waitpid wait on the process which pid is given as argument.
                If $retval$ not NULL, the exit code of the process is put at address $retval$.
            \item RETURN :
                \begin{itemize}
                    \item -1 if process does not exist
                    \item -2 if process is dead
                    \item -3 if waiting for itself
                    \item 0 otherwise
                \end{itemize}
        \end{itemize}

    \item [NAME] : \textbf{Open}
        \begin{itemize}
            \item SYSNOPSIS : int Open(const char* filename)
            \item DESCRIPTION :
                Open try to open file *filename* taking into account current directory,
                returning a unique identifier
            \item RETURN :
                \begin{itemize}
                    \item -1 if file can not be opened
                    \item -2 if MAX\_OPEN\_FILES (default 10) are already opened
                    \item -3 if the file is already opened by another thread/process
                    \item id $\in [0; MAX\_OPEN\_FILES[$ a unique identifier used for future syscall
                        \end{itemize}
                \end{itemize}

            \item [NAME] : \textbf{Close}
                \begin{itemize}
                    \item SYSNOPSIS : int Close(int id)
                    \item DESCRIPTION :
                        Close try to close file with identifier *id*.
                    \item RETURN :
                        \begin{itemize}
                            \item -1 if file $id$ does not exists
                            \item 0 otherwise
                        \end{itemize}
                \end{itemize}

            \item [NAME] : \textbf{Create}
                \begin{itemize}
                    \item SYSNOPSIS : int Create(const char *filename)
                    \item DESCRIPTION :
                        Create file $filename$ taking into account current directory.
                    \item RETURN :
                        \begin{itemize}
                            \item -1 if creation failed
                            \item 0 otherwise
                        \end{itemize}
                \end{itemize}

            \item [NAME] : \textbf{Read}
                \begin{itemize}
                    \item SYSNOPSIS : int Read(int id, char *buffer, int numBytes)
                    \item DESCRIPTION :
                        Try to read $numBytes$ inside file $id$ and store result in $buffer$.
                        $buffer$ should be large enough to fit $numBytes$.
                    \item RETURN :
                        \begin{itemize}
                            \item -1 if file does not exists
                            \item other $numReadBytes$ the real number of bytes read
                        \end{itemize}
                \end{itemize}

            \item [NAME] : \textbf{Write}
                \begin{itemize}
                    \item SYSNOPSIS : int Write(int id, const char* from, int numBytes)
                    \item DESCRIPTION :
                        Try to write inside file $id$ at most $numBytes$ bytes stored in $from$
                        memory.
                    \item RETURN :
                        \begin{itemize}
                            \item -1 if the file does not exists
                            \item otherwise $numWriteBytes$ the real number of bytes
                        \end{itemize}
                \end{itemize}

            \item [NAME] : \textbf{Seek}
                \begin{itemize}
                    \item SYSNOPSIS : int Seek(int id, int position)
                    \item DESCRIPTION :
                        Move at position $position$ inside file $id$ relative to the beginning of
                        the file.
                    \item RETURN :
                        \begin{itemize}
                            \item -1 if the file does not exists
                            \item 0 otherwise
                        \end{itemize}
                \end{itemize}

            \item [NAME] : \textbf{Remove}
                \begin{itemize}
                    \item SYSNOPSIS : int Remove(const char* name)
                    \item DESCRIPTION :
                        Delete file named $name$.
                    \item RETURN :
                        \begin{itemize}
                            \item -1 if the file does not exists
                            \item -2 if the file is opened by another process
                            \item 0 otherwise
                        \end{itemize}
                \end{itemize}

            \item [NAME] : \textbf{GetCurrentDirectory}
                \begin{itemize}
                    \item SYSNOPSIS : char *GetCurrentDirectory(char *result)
                    \item DESCRIPTION :
                        Write the current process directory (absolute path) inside buffer $result$.
                    \item RETURN :
                        \begin{itemize}
                            \item address of $result$ (never fail, can be ignored)
                        \end{itemize}
                \end{itemize}

            \item [NAME] : \textbf{SetCurrentDirectory}
                \begin{itemize}
                    \item SYSNOPSIS : int SetCurrentDirectory(const char* dirname)
                    \item DESCRIPTION :
                        Set the current directory to $dirname$ of current process.
                        $dirname$ can be relative path to current directory.
                    \item RETURN :
                        \begin{itemize}
                            \item -1 if dirname does not exists
                            \item 0 otherwise
                        \end{itemize}
                \end{itemize}
        \end{description}


\subsection{Malloc}
  An implement of malloc/free as been provided as on UNIX. realloc and calloc
  are also available.

  Same rules and same prototypes as for UNIX version (see man malloc). Our
  version dynamically allocate and release pages for heap.

  In context of multithreads, you need to call memory\_init into the main thread
  before forking thread to initialize synchronization structure. This is not
  needed (but not wrong) for no-thread applications (only the main thread).

  Three allocations strategy has been implemented :
  \begin{itemize}
  \item FIRST\_FIT
  \item BEST\_FIT
  \item WORSE\_FIT
  \end{itemize}
  To use this library, include at top of your program :

  \begin{lstlisting}[language=C]
    #define BEST_FIT 1 
    #include "mem_alloc.c"
  \end{lstlisting}
  By default, the strategy is FIRST\_FIT and the first line (define) is
  unnecessary. The mem\_alloc.*c* is needed as Makefile only compile one file as
  a binary.
